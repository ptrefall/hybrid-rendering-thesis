
\part{Introduction}

In the modern age of computer graphics, the call for techniques to achieve photo-realistic images for real-time graphics is ever increasing. Realistic lighting and shadowing of scenes, dynamic reflection and refraction, smooth silluettes, caustics, and high geometric detail at any distance are some of the problems.

For a long time, rasterization has been the choice for real-time graphics, while ray-tracing has been the superior technique for photo-realistic offline rendering.

In this paper we propose using a blend of rasterization and raytracing for real-time graphics, and try to identify the strengths and weaknesses of these different rendering techniques.

We'll look at the deferred shading technique for rasterization, allowing the entire pipeline to be represented in view space, complimenting the raytracer.

First we review previous techniques, before we look a bit closer at the rasterization and raytracer pipelines. After that we should be ready to analyze the question, \emph{why hybrid?}, and get into the complications of interopability between the rasterizer and raytracer. Next we'll go through the implementation details of our work, before we round off with a conclusion and suggestions for further work on the topic.