
\section{OptiX and GL interopability}


Combining our deferred raster pipeline with OptiX can either be done from scratch, learning what every OptiX API-call does, or, copy one of the GL/Optix interop examples and modify it until it works. 

\subsection{Shared State}

The following state must be synchronized between GL and OptiX

\begin{enumerate}
	\item Geometry and its transform (model matrix)
	\item Textures (if we find that they can demonstrate any advantages)
	\item Camera (view matrix \& projection matrix)
\end{enumerate}

For geometry, ``optix::Context'' provides functions createBufferFromGLBO and similar createTextureSamplerFromGLImage for textures.

The standard pinhole\_camera.cu implementation is used, then the view vectors (u,v,w) must be extracted from our camera class or the modelview matrix.

\paragraph{Description of the rendering pipeline}

GL and OptiX scenes are independently rendered. We end up with two sets of g-buffers, shading and composition is done in a shader.

