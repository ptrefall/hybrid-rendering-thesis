\section{OptiX}

OptiX is a closed source raytracing engine designed by Nvidia, and currently only works on their own GPUs.

Optix can be seen as a framework for launching rays. It doesn't know anything about lights, shadows, ambient occlusion or any other rendering detail for that matter. What it does know is; rays, acceleration structures, scene representation (not a traditional scene hiearchy) and scene traversers. What makes OptiX interesting is programmability; a variety of ray tracing-based algorithms in graphics and non-graphics domains can be implemented \cite{Parker10OptiX} by creating custom programs.

\paragraph{There are eight} different types of user provided programs in OptiX. These programs are implemented using the CUDA C-language. They all operate on a single ray at a time, except for the bounds program, which is used to determine bounds for acceleration structure construction. For triangle geometry, Optix can access individual vertices of a mesh for constructing a k-dimensional-tree (kdtree) or split bounding volume hiearchy (sbvh).

\begin{enumerate}
	\item {\textbf{Ray generation} programs are the entry into the ray tracing pipeline. Input and output buffers are declares inside. You can a single raygen program as a general compute shader. The intended use is to initializes ray origin and direction and then trace into the scene from a given node. Raygen programs can be used to implement a camera model. A ray generation program will create start your camera program for a given input pixel, start the trace operation using rtTrace, and store the final color in an output buffer. Calling the client function rtContextLaunch invokes many raygen programs in parallel.}

	\item{\textbf{Exeption} programs - occurs when the system encounters problems like out of stack space, or buffer access index is out of range. Also supports user-defined exceptions that can be 	thrown from any program. Can react by printing diagnostic messages or writing specific color values to an output buffer.}

	\item{\textbf{Closest hit} programs are invoked once traversal has found the closest interseciton of a ray with scene geometry.}

	\item{\textbf{Any hit}
								allow shading to be kept seperate from geometry.
								May call rtTerminateRay and stop all traversal. Early ray termination for shadowrays and AO.
								Also useful for binary transparency by texture look up.
								Default any hit program is a no-op, often the most desired operation.}

	\item{\textbf{Miss}
								programs are called when a ray did not intersect any geometry.}



	\item {\textbf{Intersection}
								programs are needed to describe geometry. The program must at least report if and where the ray touches the object, 
								additional computations may involve normals, texture coordinates and other attributes based on hit position.
								Intersection programs allow you to trace perfect spheres, cylinders, cubes, CSG-surfaces, parametric surfaces like
								Bezier and NURBS or even fractal geometries. }
	
	\item{\textbf{Selector}
								visit programs gives on control over graph traversal.
								The program can do traversals based on data stored in a visitor programs payload and make a traversal decision based 	on that data.}

 \item{\textbf{Bounding box}
 							 programs obviously compute the bounds associated with each primtive to enable acceleration structures over any geometry.
 							 The bounding box program takes a primtive index and computes its bounds. From the client api, a node is associated a given bounds program. }
 				
\end{enumerate}
			 
\paragraph{The API}

To begin raytracing a scene ``rtContextLaunch'' is called with a given Rey Generation Program index as a parameter. From the ray generation program ``rtTrace'' is given the root scene node for traversal. During traversal, if a node a nodes associated selector and acceleration program is run.

\par
If the acceleration traverser determines that a primitive node is hit, it invokes the primtives intersection program.

\par
If the Intersection program calles ``rtReportIntersection'', the primitives associated ``material'' Any Hit Program is called.

\par
If the ray doesn't hit anything, a miss program is called. The missprogram is allways required.

Scene Traversal
\begin{itemize}
\item Selector Visit program
\item ntersection program (hit geometry, time to run shade program)
\end{itemize}

Shading programs
\begin{itemize}
	\item Miss program
  \item Closest Hit program
  \item Any hit program
\end{itemize}

\begin{verbatim}
RT_PROGRAM void pinhole_camera() {
Ray ray = PinholeCamera::makeRay( launchIndex );
UserPayload payload;
rtTrace( topObject, ray, payload );
outputBuffer[launchIndex] = payload.result;
}
\end{verbatim}
